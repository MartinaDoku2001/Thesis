\documentclass[a4paper]{sapthesis}
\title{EEG signal denoising using machine learning techniques}
\author{Martina Doku}
\IDnumber{1938629}
\course{Applied Computer Science and Artificial Intelligence}
\courseorganizer{Facoltà di Ingegneria dell’Informazione, Informatica e Statistica}
\AcademicYear{2022/23}
\advisor{Prof Danilo Avola}
\copyyear{2023}
\authoremail{doku.1938629@studenti.uniroma1.it}
\begin{document}
\maketitle
\dedication{dedication}
\begin{abstract}
This thesis aims to investigate the use of machine learning
 techniques for EEG signal denoising. The first part of the
thesis is dedicated to the introduction of the problem, the statement
of the research questions and an overview of the basic concepts.
The second part is dedicated to history and 
the state of
the art of the signal analysis, and more specifically of the EEG signal
analysis and denoising methods. The third part is dedicated to the
 machine learning techniques.
The fourth part is dedicated to the experimental results and the last part
is dedicated to the conclusions and future work.
\end{abstract}
\tableofcontents
\chapter{Introduction}
The chapter is built as follows: in the first section there is a brief
introduction of the problem, the reasons that led to the choice of the
topic and the outline of the thesis. In the second section there is the
statement of the research questions. In the third section there is a brief
description of the basic concepts.
\section{Problem statement}
What is EEG in the first place and why is it important? Electroencephalography 
(EEG) is a non-invasive technique used to measure the electrical activity of
the brain. EEG signals are still among the less explored ones in the
field of signal processing, despite their widespread use in clinical 
practice and research. In recent years, there has been a growing 
interest in developing denoising methods for EEG data to improve their 
quality and reliability.\\ \\
The main objective of this thesis is to investigate and compare different
 EEG denoising methods, and to evaluate their performance in terms of 
 signal quality, artifact removal, and preservation of underlying brain
activity. Specifically, we will focus on the most recent machine learning
techniques, such as generative adversarial networks (GANs), autoencoders
(AEs), and deep learning (DL) models. We will also investigate the 
potential of combining different denoising methods to improve the 
performance of EEG denoising. \\ \\
Overall, this thesis aims to provide a better understanding of the strengths
 and limitations of different EEG denoising methods, and to help researchers
  and clinicians make informed decisions when selecting the most appropriate
   denoising method for their EEG data analysis. By improving the quality of
    EEG signals, we can enhance our understanding of brain function and 
    ultimately contribute to the development of more effective diagnostic 
    and therapeutic tools for neurological disorders.

\section{Research questions}
The research questions are the following:
\begin{itemize}
\item How can we remove artifacts from EEG signals?
\item How can we exploit the most recent machine learning techniques for
    EEG denoising?
\item What are the strengths and limitations of these new methods?
\item What are the performance of these new methods?
\end{itemize}
\section{Basic concepts}
\subsection{EEG}
When talking about EEG,in this thesis, we are referring to the
 electroencephalogram, in particular we are interested in the 
 EEG waves. Electroencephalogram (EEG) waves are the patterns 
 of electrical activity that are recorded by EEG measurements. 
 These waves have different frequencies and amplitudes, and they 
 reveal different states of brain activity. We divide the EEG waves in 
 5 main categories depending on their frquency:Alpha, Beta, Theta, 
 Delta and Gamma waves.


 Alpha waves are typically observed when someone is relaxed and awake,
  with a frequency of 8-13 Hz. Beta waves, on the other hand, are 
  associated with active cognitive processing and have a higher frequency 
  of 14-30 Hz. Theta waves are usually observed during drowsiness or
   light sleep and have a frequency of 4-7 Hz, while delta waves are
    typically observed during deep sleep and have a frequency of less 
    than 4 Hz.
 
 Gamma waves have a frequency of 30-100 Hz and are associated with higher 
 cognitive functions such as attention and memory. Mu waves, with a 
 frequency of 8-13 Hz, are observed in the sensorimotor cortex during 
 movement and motor planning.
 
 It's important to remember that EEG waves are not distinct entities but 
 represent a continuous spectrum of activity that can be influenced by
  various factors such as task demands, attention, and emotion. Interpreting EEG waves requires expertise 
  and context since different patterns of EEG activity may reflect 
  different states of brain activity depending on the individual and the 
  experimental conditions. Furthermore, research has shown that EEG waves
   can be useful in clinical diagnosis and prognosis, as well as in the
    assessment of cognitive function and brain injury. Therefore, 
    understanding the various EEG waves and their characteristics 
    can provide valuable insights into brain function and activity.
\subsection{EEG denoising}
EEG signals can be influenced by various factor that alter the real waves 
originated from neural activities, those factors are defined as artifacts. 
The artifacts can be classified as \cite{EEG artifact}: 
\begin{itemize}
    \item intrinsic artifacts:
    artifacts that depend on physiological sources, such as ocular artifacts (EOG)
    that come from eye movement and blinking, muscle artifacts (EMG) and
     cardiac artifacts (ECG)
    \item extrinsic artifacts: artifacts generated from external electromagnetic 
    such as power line noise
    sources.
\end{itemize}
Denoising of EEG data is an essential task to be able to work on data and to
extract meaningful information from it. The denoising process is complex and
it does lead to different level of quality of the data depending on the 
method used, the quality of the data and the type of artifact.\\ \\
There are several challenges \cite{denoising challenges} related both 
to single methods characteristic and general artifact removal.
For example, some methods are computationally expensive and require a lot
of time to be applied, some methods are not able to remove all kind of
artifacts, some methods require a lot of data to be applied. On a general
level, there is the problem of the lack of a standard method to evaluate
the quality of the denoised data and the EEG applications are not 
yet fuly commercial, so there hasn't been a sufficient investment in
hardware and software to make the denoising process easier.\\ \\
However the main goal of latest studies is to find a method that can
denoise from all kind of artifacts and that can be used in a flexible
and fast way, to accomodate the needs of all the different EEG applications.
\chapter{Literature review}
\section{Signal analysis}
\section{EEG denoising}
\section{Machine learning}
\chapter{Methodology}
\section{Data}
\section{Preprocessing}
\section{Denoising}
\section{Machine learning}
\chapter{Experimental results}
\section{Results}
\section{Discussion}
\chapter{Conclusions and future work}
\section{Conclusions}
\section{Future work}
\chapter{Bibliography}
\begin{thebibliography}{1}
\bibitem{EEG artifact}{iang, X.; Bian, G.-B.; Tian, Z. Removal of Artifacts from EEG Signals: A Review. Sensors 2019, 19, 987.}
\bibitem{denoising challenges}{Wajid Mumtaz, Suleman Rasheed, Alina Irfan, Review of challenges associated with the EEG artifact removal methods, Biomedical Signal Processing and Control, Volume 68, 2021, 102741, ISSN 1746-8094}
\end{thebibliography}
\chapter{Appendix}



\end{document}
