\documentclass[english]{beamer}
\usepackage[english]{babel}
\usepackage[utf8]{inputenx}
\usepackage[T1]{fontenc}      % Font encoding
\usepackage{lmodern}          % lmodern font, correctly copyable characters in pdf

\usetheme[
  bullet=circle,                  % Use circles instead of squares for bullets
  titleline=false,                % Show a line below the frame
  alternativetitlepage=true,      % Use the fancy title
  titlepagelogo=logo-sapienza,    % Logo for the first slide
  watermark=watermark-diag,   % Watermark used in every slide
  watermarkheight=20px,           % Desired height of the watermark
  watermarkheightmult=6,          % Watermark image is actually x times bigger
  displayauthoronfooter=true,     % Display author name in the footer
]{Roma}
\watermarkoff
\author{Martina Doku}
\title{Advancements in EEG Denoising and Artifact \newline Removal using Transformers}
\institute{Bachelor's degree in\\Applied Computer Science and Artificial Intelligence\\Sapienza, University of Rome}
\date{July 2023}

\begin{document}

\begin{frame}[t,plain]
\titlepage
\end{frame}

\section{Topic}
\begin{frame}{EEG}
\begin{itemize}
    \item EEG stands for Electroencephalography \newline
    \item It is a non-invasive technique used to record the electrical activity of the brain \newline
    \item It measures voltage fluctuations resulting from ionic current flows within the neurons of the brain \newline
    \item It is widely used in neuroscience for diagnosing and monitoring brain disorders and brain injuries
\end{itemize}
\end{frame}

\section{Problem}
\begin{frame}{Noise}
	\begin{columns}
	    
	    \begin{column}{0.5\textwidth}
	      Developed in 2015 by Facebook's researches, Faster-RCNN is still today an industry standard thanks to it's accuracy and performance, getting a step closer to real time object detection
	    \end{column}
	
	    \begin{column}{0.5\textwidth}
	      \begin{figure}
	        \centering
	         \caption{Faster-RCNN architecture.}
	        \end{figure}
	    \end{column}
	  \end{columns}
\end{frame}

\begin{frame}{Objective}

  \begin{columns}
    
    \begin{column}{0.5\textwidth}
      Minecraft has several desirable qualities:
      \begin{itemize}
        \item Simple graphics.
        \item Sandbox.
        \item Available to every team member.
        \item Distinguishable entity silhouettes.
      \end{itemize}
    \end{column}

    \begin{column}{0.5\textwidth}
      \begin{figure}
        \centering
       \caption{A Minecraft promotional image.}
        \end{figure}
    \end{column}

  \end{columns}

\end{frame}

\section{Solutions}
\begin{frame}{Approaches}
  \begin{columns}
    \begin{column}{0.5\textwidth}
      4000 images spread across 40 videos!

      How did we collect these videos?
      \begin{itemize}
        \item 1 minute long (circa).
        \item As many biomes as possible.
        \item One mob per video (except test).
      \end{itemize}
    \end{column}

    \begin{column}{0.5\textwidth}
      \begin{figure}[h]
          \centering
          \caption{A representative chunk of our dataset}
      \end{figure}
    \end{column}

  \end{columns}
\end{frame}

\begin{frame}{Problems}

\end{frame}
\section{Proposed Solution}
\begin{frame}{Model}
  
\end{frame}
\begin{frame}{Architecture}
  
\end{frame}
\begin{frame}{Architecture}
  
\end{frame}

\section{Results}
\begin{frame}{Comparison}

\end{frame}

\begin{frame}{Results}

\end{frame}

\section{Conclusions}
\begin{frame}{Conclusions}

\end{frame}

\begin{frame}{Greetings}

\end{frame}


\end{document}